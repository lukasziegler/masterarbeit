
\section*{Zusammenfassung}

	In den letzten Jahren haben sich Public Displays (PD) in der {\"O}ffentlichkeit stark vermehrt und wurden Teil unseres t{\"a}glichen Lebens. In Einkaufszentren, Bahnh{\"o}fen und Flugh{\"a}fen gibt es immer mehr interaktive Anwendungen f{\"u}r PDs. Ihre Entwicklung erfordert eine umfassende Evaluation, was ein komplexes und zeitintensives Unterfangen ist. Bisher greifen viele der interaktiven Anwendungen noch nicht auf die M{\"o}glichkeit zur{\"u}ck, den R{\"u}ckkanal vom PD zum Display-Anbieter zu nutzen. Um dieses Problem zu l{\"o}sen wurde eine interaktive Umfrage-Plattform entwickelt und eine umfassende Literaturrecherche durchgef{\"u}hrt. \textit{PDSurvey} soll die Durchf{\"u}hrung von Umfragen auf Public Displays erleichtern und als Werkzeug zur weiteren Evaluierung dienen. In dieser Arbeit wird der Entwurf und die Entwicklung unserer Plattform vorgestellt und eine Liste an standardisierten Frageb{\"o}gen vorgeschlagen, welche aus einer umfangreichen Literaturrecherche resultieren. Au{\ss}erdem stellen wir die Ergebnisse unserer Feldstudie vor, in der wir untersucht haben wie Umfragen auf Public Displays wahrgenommen werden und welcher R{\"u}ckkanal am besten f{\"u}r den Nutzer geeignet ist um in digitaler Form auf einen Fragebogen zu antworten.
	Die Ergebnisse lassen folgern, dass eine Mehrheit der Nutzer es vorzieht Umfragen direkt vor Ort zu beantworten. Allerdings hat auch ein Viertel sich gegen die M{\"o}glichkeit entschieden, direkt vor Ort auf den Fragebogen zu antworten. Ein Tablet als R{\"u}ckkanal anzubieten hat sich als beste Option herausgestellt, auch wenn der Benutzer zwischen den Ger{\"a}ten wechseln muss. Umfragen welche direkt auf PDs durchgef{\"u}hrt werden stellen eine sinnvolle Alternative zu Online-Umfragen dar, mit der Einschr{\"a}nkung der sozialen Erw{\"u}nschtheit und der Abnahme der Privatsph{\"a}re.


\selectlanguage{english}
\section*{Abstract}

	In recent years, public displays (PD) have proliferated in public space and become part of our daily lives. New interactive applications for PDs are flourishing in shopping malls, train stations, and airports. Their development requires extensive evaluation, which is a complex and time intensive endeavor. So far, many interactive PDs still lack a feedback channel from display to display provider. To solve this problem an interactive survey platform was developed and an extensive literature review carried out.
	\textit{PDSurvey} aims to facilitate the execution of surveys on public displays and is a toolset for further PD evaluation. In this thesis, the design and development process of our platform is presented and a list of standardized questionnaires proposed, resulting from an extensive literature review. Furthermore, we present the findings of our field study, in which we assessed the general acceptance of questionnaires being conducted in public space and which feedback channels are best suited for users to respond to questionnaires in a digital form.
	The findings imply that a majority of users prefer to complete a survey directly on-site. However, a quarter refrained from using PDs for responding to the questionnaire. Offering the tablet as a feedback channel represented the best choice, even though users have to switch devices. Surveys conducted on public displays are a reasonable alternative to online surveys, with the limitation of social desirability and a decrease in privacy.
	


	% BENEFITS: simplifying evaluation, better scalability, offering a feedback channel, comparison of PDs based on their context

	% potential LIMITATIONS: social desirability, less in-depth responses, decrease of privacy


	% Inspiration for good Abstracts: MyPosition \cite{valkanova2014myposition}, Mueller 2010: Requirements and Design Space for Interactive Public Displays \cite{muller2010requirements}, 