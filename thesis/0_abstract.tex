
\section*{Zusammenfassung}

	Kurzzusammenfassung der Arbeit, maximal 250 W\"orter.

\selectlanguage{english}
\section*{Abstract}

	In recent years public displays (PD) have proliferated in public space and are becoming part of our daily lives. In this thesis we present the design and development process of \textit{PDSurvey}, an interactive survey platform for public displays. Additionally we propose a list of standardized questionnaires, resulting from an extensive literature review. PDSurvey aims to facilitate the conduction of surveys on public displays and is a toolset for further PD evaluation. Furthermore we present the findings of our field study, in which we assess the general acceptance of questionnaires being conducted in public space and assess which feedback channels are best suited for users to respond to questionnaires in a digital form.
	Our findings are that three-fourths of our users preferred to complete the survey directly on-site, nonetheless a quarter refrained from using a PD for responding to the questionnaire. Interestingly the tablet option was by far the most popular feedback channel, although it required users to switch devices, since they first used the TV screen. These findings were reinforced with semi-structured interviews conducted in the field study.
	It can be concluded that surveys conducted on public displays are a good alternative to online surveys, with the limitation of social desirability and a decrease of privacy.

	% what follows from my solution?
	% check: did I convey why this is an interesting problem?







	% - - - - - - - - - - - - - - - - -

\section*{Draft}

	The feedback received from the field study motivated us to further improve the platform and will be ...

	Four feedback channels were offered for response: on a TV screen, on a tablet, via smartphone, or via email.

	> conducting surveys on public displays is a solid path and has opportunity for more research.

	% Furthermore we present the findings of our field study, in which we assess the general acceptance of questionnaires being conducted in public space and assess which feedback channels are best suited for users to respond to questionnaires in a digital form.

	% We evaluate PDSurvey regarding which feedback channel is best suited for response, and what the motivation for approaching was.


\textbf{Inspiration}

	\begin{itemize}
	% General introduction
	\item Interactive screens in public are already deployed in urban environments, such as airports, shopping malls, train stations, 
	\item This new form of XXX, not only beings a new way of multimedia presentation with it, but also enables a new variety of interactive applications and a new way of collecting quantitative and qualitative data.
	\end{itemize}

	A shorter way, inspired by MyPosition \cite{valkanova2014myposition}:
	\begin{enumerate}
	\item We \textbf{present} the design and development process of \textit{PDSurvey}, an interactive survey platform for public displays. Additionally we propose a list of standardized questionnaires, useful for public display evaluation.
	\item PDSurvey aims to facilitate the conduction of surveys on public displays...
	\item We \textbf{evaluated} PDSurvey regarding two first research questions, <feedback channel, (question type), motivation for approaching>.
	\item We \textbf{found} that most users prefer to complete survey directly on-site.
	\end{enumerate}

	Other papers
	\begin{itemize}
	\item Mueller 2010: Requirements and Design Space for Interactive Public Displays \cite{muller2010requirements}.
	\item MyPosition \cite{valkanova2014myposition}
	\end{itemize}