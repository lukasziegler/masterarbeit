
\section*{Zusammenfassung}

	Kurzzusammenfassung der Arbeit, maximal 250 W\"orter.

\selectlanguage{english}
\section*{Abstract}

	In recent years public displays (PD) have proliferated in public space and are becoming part of our daily lives. New interactive applications for PDs are flourishing in shopping malls, train stations, and airports. Their development requires extensive evaluation, being a complex and time intensive endeavor itself. Getting feedback on metrics such as usability involves high effort. So far, many interactive PDs are still lacking a feedback channel from the display to the display provider. To solve this problem we developed an interactive survey platform and carried out an extensive literature review.
	\textit{PDSurvey} aims to facilitate the conduction of surveys on public displays and is a toolset for further PD evaluation. In this thesis we present the design and development process of our platform and propose a list of standardized questionnaires, resulting from an extensive literature review. Furthermore we present the findings of our field study, in which we assessed the general acceptance of questionnaires being conducted in public space and which feedback channels are best suited for users to respond to questionnaires in a digital form.
	Our findings imply that the majority of users preferred to complete a survey directly on-site, nonetheless around a quarter refrained from using PDs for responding to the questionnaire. Offering the tablet as a feedback channel represented a good alternative, even though users have to switch devices. Semi-structured interviews have shown that our findings are consistent with the experience of users from the field study.

	Surveys conducted on public displays are a reasonable alternative to online surveys, with the limitation of social desirability and a decrease of privacy.

	\textbf{TODO: leicht kuerzen, aktuell sind es 260 Woerter!}
	
	% Ende ggf. ausführlicher schreiben (1-2 Sätze mehr schreiben)

	\textbf{Alternative:} Surveys conducted on public displays are a reasonable alternative to online surveys, with the benefits of getting feedback right in place, reducing the workload and allowing for a comparison of PDs based on their context. Potential limitations of conducting surveys on PDs are social desirability, due to the public setting, less in-depth responses and a decrease of privacy.


	% benefits: simplifying evaluation, better scalability, offering a feedback channel, comparison of PDs based on their context

	% potential limitations: social desirability, less in-depth responses, decrease of privacy






	% - - - - - - - - - - - - - - - - -

\textbf{Draft}

	The feedback received from the field study motivated us to further improve the platform and will be ...

	Four feedback channels were offered for response: on a TV screen, on a tablet, via smartphone, or via email.

	> conducting surveys on public displays is a solid path and has opportunity for more research.

	% Furthermore we present the findings of our field study, in which we assess the general acceptance of questionnaires being conducted in public space and assess which feedback channels are best suited for users to respond to questionnaires in a digital form.

	% We evaluate PDSurvey regarding which feedback channel is best suited for response, and what the motivation for approaching was.


\textbf{Inspiration}

	\begin{itemize}
	% General introduction
	\item Interactive screens in public are already deployed in urban environments, such as airports, shopping malls, train stations, 
	\item This new form of XXX, not only beings a new way of multimedia presentation with it, but also enables a new variety of interactive applications and a new way of collecting quantitative and qualitative data.
	\end{itemize}

	A shorter way, inspired by MyPosition \cite{valkanova2014myposition}:
	\begin{enumerate}
	\item We \textbf{present} the design and development process of \textit{PDSurvey}, an interactive survey platform for public displays. Additionally we propose a list of standardized questionnaires, useful for public display evaluation.
	\item PDSurvey aims to facilitate the conduction of surveys on public displays...
	\item We \textbf{evaluated} PDSurvey regarding two first research questions, <feedback channel, (question type), motivation for approaching>.
	\item We \textbf{found} that most users prefer to complete survey directly on-site.
	\end{enumerate}

	Other papers
	\begin{itemize}
	\item Mueller 2010: Requirements and Design Space for Interactive Public Displays \cite{muller2010requirements}.
	\item MyPosition \cite{valkanova2014myposition}
	\end{itemize}


\textbf{alter Stand}

	In recent years public displays (PD) have proliferated in public space and are becoming part of our daily lives. New interactive applications for PDs are flourishing in shopping malls, train stations, and airports. Their evaluation however is a time and cost intensive endeavor. 
	Getting feedback on metrics such as usability involves high commitment and requires a lot of effort.

	Interestingly the tablet option was by far the most popular feedback channel, although it required users to switch devices. 

