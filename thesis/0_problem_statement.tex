\section*{Aufgabenstellung} % section* = kein Eintrag ins Inhaltsverzeichnis
\label{aufgabenstellung}

\textbf{Development of a Public Display Survey Platform}

\begin{description}
    \item[Problem Statement] Public displays are quickly proliferating in public spaces. At the same time, interactive applications are still scarce, since their development is costly and the effect on the user - and thus their benefit - is often not clear. Hence, interactive display applications are usually developed, deployed, and carefully evaluated in research contexts. In most cases, evaluation focuses on particular aspects only, such as user performance, user experience, or social implications, due to the significant effort associated with planning, preparing and conducting public display evaluations.
    \item[Scope of the Thesis] To tackle the aforementioned challenge, the objective of this thesis is to develop a survey tool that allows interactive public display installations to be comprehensively assessed. As a first step, an extensive literature review will be conducted with the aim of identifying important aspects of public display deployments - from a researchers' as well as from a practitioners' perspective - as well as to develop an understanding of how these aspects could be addressed through surveys. Based on the literature review, a web-based survey platform will be implemented that can easily be used to evaluate and compare public displays through different channels. Such channels include evaluation directly at the display or through a (mobile) website that allows participation also via a smartphone or tablet. The platform should allow public display owners to configure their own surveys based on their needs. Optionally, the survey tool itself will be evaluated with an interactive public display application.

    \item[Tasks] 
    (1) conduct a literature review to identify (research) questions that are of interest to researchers and practitioners \newline
(2) produce a comprehensive set of questions that can be used to assess these questions by means of a survey \newline
(3) develop a web-based public display survey platform consisting of (a) an administration interface that allows (groups of) questions to be selected for use within the tool and
(b) a responsive UI that can be rendered on different devices (public display, smartphone, tablet, laptop)

    \item[Requirements] Strong skills in web programming, independent scientific work and creative problem solving, experience in creating questionnaires is a plus.

    \item[Keywords] Public displays, interaction, applications, survey, questionnaires, web

\end{description}