\section{Introduction}
\label{sec:introduction}


\subsection{Outline}

  
	\begin{enumerate}

	\item Introduction to the topic: there seems to be a clear trend towards more public displays and an increasing ratio of ...

	\item \textbf{Context and Goal of Thesis}: State the goal of the paper: why the study was undertaken / the paper was written. Do not repeat the abstract. 

	\item Sufficient \textbf{background information} to allow the reader to understand the context and significance of the question you are trying to address. 

	\item \textit{not sure about this one...}: Proper acknowledgement of the \textit{previous work} on which you are building. Sufficient references such that a reader could, by going to the library, achieve a sophisticated understanding of the context and significance of the question.  >> doesn't this belong in the next chapter?


	\item \textbf{Motivate} your reader to read the rest of the thesis. You should draw the reader in and make them want to read the rest of the paper. An important/interesting scientific problem that your paper either solves or addresses.

	\item Already state the \textbf{research question} here!?!? I would say so, because intrdocuding it in chapter 4 is a little to late. How about stating the general research question here (``How to automatically evaluate public display setups via surveys carried out on the displays themselves.''), and in chapter 4 getting more specific and talking about the 1) feedback channel, 2) question types, and 3)motivation for approaching the display.

	\item \textbf{Give Overview}: A verbal \textit{road map} or verbal \textit{table of contents} guiding the reader to what lies ahead. 

	\item \textbf{where??}: Explain the scope of your work, what will and will not be included. 
	\end{enumerate}




Important things to keep in mind!

  \begin{enumerate}
  \item Make sure that it is obvious where the introductory material (the \textit{old stuff}) ends and where your contribution (\textit{new stuff}) starts?
  \end{enumerate}






\subsection{First draft (notes)}

	Scope for the practical part of the thesis. What is the system supposed to look like. What is the current state of the practical work in research and in the industry. Which partners do we have, where do we want to deploy this system. What are the goals for this evaluation plattform?

	% + motivate why we built this platform, why is it needed for our research question?


	\begin{enumerate}
	\item Describe the increase of interactive displays in public, get the shift to airports, shopping malls, universities. Growing research area.
	
	\item What is the current situation like?

	\item 
	\end{enumerate}





\subsection{not sure where to mention it}

	\begin{enumerate}
	\item What are public displays?
	\item Survey vs Questionnaire
	\end{enumerate}


	\subsection{Research Question}



	\subsection{Approach}



	Benefits of PDSurvey plattform:

	\begin{itemize}
	\item being able to work with big data. collecting a large number of responses from a variety of displays in various settings, and assigning a specific context to every display connected to PDSurvey. Once enough data is collected, having the ability to evaluate and compare the displays between each other. Interesting questions for analysis would be, which role the context plays on how the users behave, when running identical software settings on the displays, but only varying the context (position, size of display, surrounding environment of the display, positioning it outdoors or indoors, influence of the weather, type of building it is positioned in) -- see Chapter 4: Modeling
	\end{itemize}


	\subsection{Overview}



	\subsection{Our Contribution}

	\begin{enumerate}
	\item categorization of questionnaires being useful for the evaluation through automated public survey display platforms
	\item research platform

	\end{enumerate}








	\subsection{Other introductions for inspiration}

	\begin{enumerate}
	\item paper 60 mueller2010mm - Requirements and Design Space for  Interactive Public Displays: Good introduction from Mueller et al. \cite{muller2010requirements}: ``Digital immersion is moving into public space. [...] analogies to HCI (utility, usability, likeability) often hold true, but not always! These need to be evaluated individually.''

	\item Another good introduction as an inspiration: ``MyPosition: Sparking Civic Discourse by a Public Interactive Poll Visualization'' >> check out the introduction there! \cite{valkanova2014myposition}

	\item paper 18 cites Mark Weiser, on his predictions regarding Ubiquitous Computing and how it will be the third revolution after main frame (1st) and personal computer (2nd). for the quote see GDrive `Quotes and Notes from Literature'

	\item copy further notes from https://docs.google.com/document/d/1f2MJHMh5Yvvh9d4hhIT0WmgdlBbZ9rX5bG8dNkS50vg/edit
	\end{enumerate}



