\section{Introduction}
\label{sec:introduction}



\begin{enumerate}
\item nice introduction in paper: valkanova2014myposition
\end{enumerate}



\subsection{Motivation}

Scope for the practical part of the thesis. What is the system supposed to look like. What is the current state of the practical work in research and in the industry. Which partners do we have, where do we want to deploy this system. What are the goals for this evaluation plattform?


  \begin{enumerate}
  \item What are public displays?
  \item 
  \end{enumerate}


\subsection{Research Question}



\subsection{Approach}



\subsection{Overview}



\subsection{Our Contribution}

\begin{enumerate}
\item categorization of questionnaires being useful for the evaluation through automated public survey display platforms

\end{enumerate}




% EXAMPLE TEXT

% \begin{enumerate}[itemsep=0pt] 
% \item option 1
% \item option 2
% \end{enumerate}


%\begin{figure}%[btph]
  %% Datei ``beispielbild.eps'' oder ``beispielbild.png'', zentriert
  %\begin{center}\includegraphics{beispielbild}\end{center}

  %% Datei auf 8cm Breite verkleinert/vergr��ert
  %\includegraphics[width=8cm]{beispielbild}
  %% Datei auf ganze Breite des Texts vergr��ert
  %\includegraphics[width=\columnwidth]{beispielbild}
  %% Datei auf 60% der Textbreite verkleinert/vergr��ert
  %\includegraphics[width=.6\columnwidth]{beispielbild}
  %% Weitere Optionen (Ausschnitt, drehen etc.) in der Doku zum graphicx-Paket

%  \begin{center}\LARGE [BILD]\end{center}
%  \caption{Bildunterschrift}
%  \label{fig:beispielbild}
%\end{figure}

    %Siehe Abbildung \ref{fig:beispielbild} oder einschl\"agige Literatur, z.B.
    %\cite[Seite 6]{Ivory01} oder \cite{NielsenAlertbox}.


% \begin{figure}
%   \begin{center}\LARGE [BILD]\end{center}
%   \caption{Bild}
%   \label{fig:beispielbild3}
% \end{figure}

