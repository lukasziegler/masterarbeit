\section{Introduction}
\label{sec:introduction}


%% GOOD INTRODUCTION, see ...

  % paper 60 mueller2010mm - Requirements and Design Space for  Interactive Public Displays:

  Good introduction from Mueller et al. \cite{muller2010requirements}: ``Digital immersion is moving into public space. [...] analogies to HCI (utility, usability, likeability) often hold true, but not always! These need to be evaluated individually.''

  Another good inspiration from a Mueller et al. paper: ``MyPosition: Sparking Civic Discourse by a Public Interactive Poll Visualization'' >> check out the introduction there!



\begin{enumerate}
\item nice introduction in paper: valkanova2014myposition
\item paper 18 cites Mark Weiser, on his predictions regarding Ubiquitous Computing and how it will be the third revolution after main frame (1st) and personal computer (2nd). for the quote see GDrive `Quotes and Notes from Literature'
\item copy further notes from https://docs.google.com/document/d/1f2MJHMh5Yvvh9d4hhIT0WmgdlBbZ9rX5bG8dNkS50vg/edit#
\end{enumerate}



\subsection{Motivation}

Scope for the practical part of the thesis. What is the system supposed to look like. What is the current state of the practical work in research and in the industry. Which partners do we have, where do we want to deploy this system. What are the goals for this evaluation plattform?

% + motivate why we built this platform, why is it needed for our research question?


  \begin{enumerate}
  \item What are public displays?
  \item Survey vs Questionnaire
  \end{enumerate}


\subsection{Research Question}



\subsection{Approach}



Benefits of PDSurvey plattform:

\begin{itemize}
\item being able to work with big data. collecting a large number of responses from a variety of displays in various settings, and assigning a specific context to every display connected to PDSurvey. Once enough data is collected, having the ability to evaluate and compare the displays between each other. Interesting questions for analysis would be, which role the context plays on how the users behave, when running identical software settings on the displays, but only varying the context (position, size of display, surrounding environment of the display, positioning it outdoors or indoors, influence of the weather, type of building it is positioned in) -- see Chapter 4: Modeling
\end{itemize}


\subsection{Overview}



\subsection{Our Contribution}

\begin{enumerate}
\item categorization of questionnaires being useful for the evaluation through automated public survey display platforms

\end{enumerate}




% EXAMPLE TEXT

% \begin{enumerate}[itemsep=0pt] 
% \item option 1
% \item option 2
% \end{enumerate}


%\begin{figure}%[btph]
  %% Datei ``beispielbild.eps'' oder ``beispielbild.png'', zentriert
  %\begin{center}\includegraphics{beispielbild}\end{center}

  %% Datei auf 8cm Breite verkleinert/vergr��ert
  %\includegraphics[width=8cm]{beispielbild}
  %% Datei auf ganze Breite des Texts vergr��ert
  %\includegraphics[width=\columnwidth]{beispielbild}
  %% Datei auf 60% der Textbreite verkleinert/vergr��ert
  %\includegraphics[width=.6\columnwidth]{beispielbild}
  %% Weitere Optionen (Ausschnitt, drehen etc.) in der Doku zum graphicx-Paket

%  \begin{center}\LARGE [BILD]\end{center}
%  \caption{Bildunterschrift}
%  \label{fig:beispielbild}
%\end{figure}

    %Siehe Abbildung \ref{fig:beispielbild} oder einschl\"agige Literatur, z.B.
    %\cite[Seite 6]{Ivory01} oder \cite{NielsenAlertbox}.


% \begin{figure}
%   \begin{center}\LARGE [BILD]\end{center}
%   \caption{Bild}
%   \label{fig:beispielbild3}
% \end{figure}

