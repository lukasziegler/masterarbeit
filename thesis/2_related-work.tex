\section{Related Work}
\label{chapter:related-work}

	% Brief introduction
	The goal of the literature review was to find out how other researchers evaluated public displays and to develop an understanding of how these aspects can be addressed through surveys. The aim was to identify important aspects of public display deployments, both from a researcher's as well form a practitioner's perspective. A short summary of the most relevant papers are described in the following.


	\subsection{Evaluation of Public Displays} % Previous evaluation

	Public display evaluation has already been addressed in literature. Alt et al. \cite{Alt2012HowToEvaluate} give an overview of study types, research paradigms, and evaluation methods used for evaluating public displays. M{\"u}ller et al. \cite{muller2014mirrortouch} present with MirrorTouch a follow-up evaluation and additionally extract metrics used for quantitative field studies. According to their findings almost exclusively descriptive field studies are used in the area of public display evaluation. For a more in-depth introduction to public displays, the doctoral thesis from Alt \cite{alt2013thesis} gives a good overview.
	% General introduction to Evaluation 
	For a general recap of how to best design, evaluate, and report experiments, the book by Field and Hole \cite{field2003design} was used. Kirakowski \cite{kirakowski2000questionnaireFAQ} provides a good introduction for practitioners of what to watch out for, covering the most important aspects for conducting surveys.




	\subsection{Exemplary Papers}

	A selection of papers, which have inspired us and which themselves have a good approach towards the evaluation of public displays, are amongst others: Overcoming Assumptions by Huang et al. \cite{huang2008overcoming}, Worlds of Information by Jacucci et al. \cite{jacucci2010worldsofinformation}, and Digifieds by Alt et al. \cite{alt2011digifieds}. In the following a short overview is given.


	% Worlds of Information - 2010
	Jacucci et al. \cite{jacucci2010worldsofinformation}, with \textit{Worlds of Information}, not only provide a superb overview of evaluation methods, but also the way they evaluate their results is exemplary. The evaluation is detailed and the questions asked in their questionnaires are stated. For evaluation they used video ethnography and a variety of questionnaires. Their focus lies in their paper lies on aspects of group use and user engagement, which they measure through questions adapted from Flow (GameFlow), Presence (MEC), and Intrinsic Motivation (IMI) questionnaires.

	% Overvcoming Assumptions (Huang) - 2008
	In \textit{Overcoming Assumptions} Huang et al. \cite{huang2008overcoming} analyzed how ``large ambient information displays in public settings'' (public displays) were deployed and tried to understand how such displays are used, in order to derive best practices and offer design implications. Recommendations given include elements like the position of the display, content, dynamics or how to best present the message you want to convey.
	% IM Here (Huang) - 2004
	In \textit{IM Here}, an older publication by Huang et al. \cite{Huang2004} observations, informal conversations and questionnaires are used for evaluation. The focus of their evaluation is social awareness and collaboration aspects. Questionnaires were conducted before and after the primary task.

	% UBI Hotspot (Ojala) - 2010 - #41
	Ojala et al. present an evaluation of their long-term public display deployment in downtown Oulu, referred to as \textit{UBI Hotspot}. They conducted their surveys in-situ and evaluated data from a time span of eight months. Surveys were also already embedded directly onto the UBI hotspots. They aimed for analyzing the usage and user acceptance rates for their setup with questions based on Nielsen's system acceptance model. Two general demographic questions (age, gender) were asked first, followed by a random selection of 8 statements from Nielsen's system acceptance model.


	% Digifieds (Alt) - 2011
	Alt et al. \cite{alt2011digifieds} created a digital noticeboard called \textit{Digifieds} and evaluated the platform using observations, interviews and a field trial. The Digifieds platform was deployed during the UbiChallenge 2011 to an urban environment in Finland and evaluated with the help of SUS questionnaires. The questionnaire accompanied the field trial and was structured as follows. After the users got a brief introduction to the study, general questions were asked regarding the mobile phone usage, whether the display setup has been used before, and how they are affiliated towards public notice areas. Thereafter, two practical tasks were carried out, each followed by a SUS questionnaire. Finally, questions were asked regarding the user's opinion on public notice areas. 


	% Looking Glass (Müller) - 2012
	M{\"u}ller et al. \cite{Muller2012LookingGlass} present an in-depth evaluation of \textit{Looking Glass}, an interactive display setup inside of a shop window. Looking Glass gave visual feedback to passers-by and mirrored their movements. Their evaluation consisted of a pre-study, a controlled lab-study, and an ``in the wild'' field study. Both quantitative and qualitative data was collected, however, no questionnaires were used. Mueller et al. relied on observations, semi-structured interviews, and manual video recording, combined with interaction logs and a depth video log for quantitative data.





	\textbf{TODO}

	% Audience Behaviour (Beyer) - 2011 - #4
	\paragraph{Beyer et al.} start off with a demographic questionnaire, a short briefing, executing the actual task and conducting an AttrakDiff questionnaire in the end.

	% Sweep and Point and Shoot (Ballagas) - 2005 - #3
	\paragraph{Ballagas 2005 \#3} used a background, subjective and a ISO 9241-9 based questionnaire, for measuring performance.

	% "The Challenge of Evaluating Situated Display-based" - 2008
	\paragraph{Cheverst \#11} Cheverst et al. - their focus is determining the \textit{sense of community} index. They first collect ethnographic and cultural data, followed by a focus group and a design workshop.
		% 2005		
		\textbf{Hermes Photo Display \#10} also focus on the community aspect, measuring the ``notions of community'' and some individual questions. However in this paper no standardized questions are used, but rather individual questions from the researchers.



%%%  TEMPORARY NOTES  %%%

		% >> introduce the reader to the area of public display evaluation, how they have been evaluated, what has to be known, and so on

		% \item state how complex it is to administer / execute / conduct a survey or questionnaire
		% \item encourage the motivation for creating a plattform like this!
		% \item discrepancy between field and lab study: \cite{Ojala2011}.
		% \item field studies are much more time consuming and usually spread over a larger area to assess. Being able to automate certain parts, such as the collection of quantitative or qualitative data, will facilitate the evaluation process for researchers and give new insights for display operators.

%%%  TEMPORARY NOTES  %%%






	\subsection{Overview of Survey Platforms}
	% Other tools, possibly related to ours.
	Before we started developing the research platform, we did some research on which similar tools there are available on market. To verify that a toolset like ours doesn't already exist and to better understand how web-based survey platforms are caried out, we checked which similar approaches already exist on the market today.

	% LMU: SosciSurvey
	The first one we looked at was SosciSurvey \footnote{\url{https://www.soscisurvey.de/} (last accessed on November 26, 2014)}, a popular tool developed by the Institute for Communication Science at our university. One major drawback of their approach was the difficulty of extending and embedding the SociSurvey platform. For our purpose it was easier to build a new platform already supporting a responsive layout and RESTful interaction. Other tools similar to the SociSurvey platform are LimeSurvey \footnote{\url{http://de.wikipedia.org/wiki/LimeSurvey} (last accessed on April 6, 2015)}, an open-source PHP project.

	% Commercial solutions
	Commercial solutions which support far more than the free platforms, are 
	
		eSurvey Creator \footnote{\url{https://www.esurveycreator.com/} (last accessed on April 6, 2015)}
		Free Online Surveys is missing some question types (e.g. the Likert scale) and freeonlinesurveys.com

		UX Suite by UsabilityTools \footnote{\url{http://usabilitytools.com/ux-suite/} (last accessed on April 6, 2015)}, has a handy backend for configuring the surveys, although it is lacking the pre-configured standardized questionnaires. 
		

		SurveyMonkey \footnote{\url{https://www.surveymonkey.com} (last accessed on April 6, 2015)} provide sample surveys\footnote{\url{https://www.surveymonkey.com/blog/en/sample-survey-questionnaire-templates/} (last accessed on April 6, 2015)} and a mobile app for conducting and evaluating surveys. Their approach already comes closer to what we are looking for. However they do not offer an API for embeding platforms with other programing languges and the standardized questionnaires are relevant to public display research.

		SoGoSurvey \footnote{\url{http://www.sogosurvey.com/Features/List-of-All-Features.aspx} (last accessed on April 6, 2015)}. One of the best solutions!

		SurveyPlanet () seems to be the best solution found so far! However the embed code is only based on iframes.

		Qualtrics \footnote{\url{http://www.qualtrics.com/site-intercept/} (last accessed on April 6, 2015)}


	The most interesting was the solution of XXX ... \textbf{TODO TODO}

	For additional information on currently available web-based survey tools, refer to \cite{Capterra2015SurveyTools, Idealware2011SurveyTools}





	\subsection{What makes our platform unique (rephrase)}
	% What is unique about our approach
	The key difference between our approach and the already existing evaluation platforms, is the ability to associate each survey with the display on which the survey is carried out on, and that the platform is tailored specifically for public display evaluation.
	By default \textit{PDSurvey} asks the display operator to specify the context of every display connected to the platform. When enough context data is specified, this will allow for a thorough evaluation and comparison of public display installations.
	Further technical differences, which are not yet standard on all of the free and commercial solutions yet, is the ability to conduct surveys across a broad number of devices and platforms, due to its modular and RESTful architecture. The benefit is that the whole platform retrieves all data via a RESTful API, as of now allowing for the greatest possible coverage of end consumer devices. Additionally we offer a range of standardized questionnaires, simplifying the evaluation of public displays. These questionnaires used for evaluation will be introduced in section \ref{section:questionnaires:categorization}.

	These measures allow for a simplified evaluation of public displays. Standardized questionnaires useful for public display evaluation are already implemented by default.

	% REST API is missing, not designed for use on public displays. Thus no overlay and seamless integration into public display applications is possible. The only exception so far might be Qualtrics, based assessing their Site Intercept demo.

	> NEW: by having a clear focus on public display evaluation, this also allows us tweak and modify the question types to better be suited for public displays. Some question types from traditional web-based questionnaires are not well-suited for large displays.



	> NEW: % Development in the last years (TODO, maybe leave it out)
	Already in 1983 a study was conducted by Sproull and Kiesler, comparing email with traditional mail surveys. Their findings were a faster and cheaper conduction of surveys via email, with less social desirability. Possible restrictions were the limited spread of computers and thus a distorted population \cite{sproull1986reducing}. In 1999 commercial companies like SurveyMonkey were founded, providing first online survey platforms. The same counts for surveys being conducted on public displays, which opens up a whole new feedback channel. With the introduction of the iPhone in 2007 and the iPad in 2010 users became much more accustomed to using screens with touch support. The increasing acceptance of touch screens combined with more sophisticated touch screen technology opens the path for interactive questionnaires in public settings. 



%%%  REMINDERS  %%%

	% The introduction should be focused on the thesis question(s). All cited work should be directly relevant to the goals of the thesis. This is not a place to summarize everything you have ever read on a subject.

	% 1. give an overview of related work
	% 2. give background information to this thesis
	% 3. describe the work of others, what have they done so far?

	% The goal of the literature review was to find out how other researchers evaluate public displays. The aim was to identify important aspects of public display deployments - both from a researcher's as well form a practitioner's perspective. Furthermore it was of interest to develop an understanding of how these aspects could be addressed through surveys. 
