\section{Related Work} % ODER "Background and Related Work"
\label{sec:related-work}

	% Brief introduction
	Our research is based on an extensive literature review, with over 100 papers viewed. This led to the categorization of standardized questionnaires (see chapter \ref{sec:questionnaires}) and the development of the public display survey platform (see chapter \ref{sec:implementation}). A short summary of the most relevant paper is described below
	The goal of the literature review was to find out how other researchers evaluated public displays and to develop an understanding of how these aspects can be addressed through surveys. The aim was to identify important aspects of public display deployments, both from a researcher's as well form a practitioner's perspective. 

	% Previous evaluation
	Public display evaluation has already been addressed in literature. Alt et al. \cite{Alt2012HowToEvaluate} cover the study types, paradigms, and evaluation methods used for evaluation of public displays.

	M{\"u}ller et al. \cite{muller2014mirrortouch} classify evaluation methods, extract metrics and give an overview of evaluation methods.

	For a more in-depth introduction to public display evaluation the doctoral thesis of Alt \cite{alt2013thesis} gives a good overview.


	 The general principles of public display evaluation have already been discussed there, on which the following research is based.

	% General introduction to Evaluation >>> For a short recap 
	For a short recap of how to best design and evaluate experiments and evaluations the book ``How to design and Report Experiments'' by Field and Hole \cite{field2003design} was used. There, the basics for evaluation can be found for a solid foundation.





	% Wie PDs bisher evaluiert wurden

	The doctoral thesis by Florian Alt \cite{alt2013thesis}

	Mueller et al. \cite{muller2014mirrortouch} give an overview of evaluation methods for public display. According to their findings almost exclusively descriptive field studies are used in the area of public display evaluation.





%%%  REMINDERS  %%%


	% The introduction should be focused on the thesis question(s).  All cited work should be directly relevent to the goals of the thesis.  This is not a place to summarize everything you have ever read on a subject.

	% 1. give an overview of related work
	% 2. give background information to this thesis
	% 3. describe the work of others, what have they done so far?

	% The goal of the literature review was to find out how other researchers evaluate public displays. The aim was to identify important aspects of public display deployments - both from a researcher's as well form a practitioner's perspective. Furthermore it was of interest to develop an understanding of how these aspects could be addressed through surveys. 




%%%  ACTUAL RELATED WORK  %%%


	%%%  1st part: Book
	\textbf{1) How to evaluate? A short recap of best-practices.}

		\begin{enumerate}
		\item How to Design and Report Experiments \cite{field2003design}
		\item explain how researchers usually proceed (quantitative, qualitative)
		\item things they have to take care of, some elements which can be optimized (quantitative analysis)


		\item Kirakowski [201 Questionnaires in Usability Engineering.pdf] - A list of FAWs: What is a questionnaire? What kind of questions are there? What kind of questionnaires are there? What are the advantages and disadvantages of using a questionnaire? 		--   A MUST READ / REFERENCE !!!



		\item Alt et al. \cite{Alt2012HowToEvaluate}  -  The publication ``How to evalaute public displays'' by Alt et al. gives a first foundation on how to evaluate public displays. \url{http://uc.inf.usi.ch/sites/all/files/ispd2012-alt.pdf} % ++ STATE 3 KEY FINDINGS ++ % 
		\item + but above all: \textit{Florian's PhD thesis} \cite{alt2013thesis}!

		\item Mueller et al. \cite{muller2014mirrortouch} give an overview of evaluation methods for public display. According to their findings almost exclusively descriptive field studies are used in the area of public display evaluation. \url{http://joergmueller.info/pdf/MHCI14MuellerMirrorTouch.pdf}

		\end{enumerate}




	\textbf{2) Overview of papers with a relevance for the construction / or with a good evaluation.}

		\begin{enumerate}
		\item Jacucci: Worlds of Information (paper 25: \url{http://www.hiit.fi/u/morrison/chi2010.pdf}) liefert einen erstklassigen Ueberblick ueber die Evaluationsmethoden
		\item Papers on which I did a meta-analysis, comparing and evaluating how they approached the evaluation via questionnaires.
	\end{enumerate}






	\textbf{3) List of Standardized Questionnaires}
		% List all the standardized questionnaires I found during my research phase.

		During the literature review phase a comprehensive list of widely used questionnaires was assessed. A list of other people's collections can be found here:

		\begin{enumerate}

		\item Jeff Sauro and James Lewis - HCI International 2013 - list of 19 questionnaires - written 2013: \url{http://2013.hci.international/index.php?module=pagesmith&uop=view_page&id=44}

		\item Usability and user experience surveys - 9 surveys	- written 2011 - 2014: \url{http://edutechwiki.unige.ch/en/Usability_and_user_experience_surveys}

		\item UX Research | Standardized Usability Questionnaires - list of SUMI, PSSUQ, SUS	written Nov 2013 - \url {http://chaione.com/ux-research-standardizing-usability-questionnaires/}
		\item Overview of Usability Questionnaires: ISONorm 110, ISOMetrics, AttrakDiff, UEQ, QUIS, SUMI - 2014 - \url{http://www.cheval-lab.ch/was-ist-usability/usabilitymethoden/frageboegen/}

		\item Overview of Usability and User Experience Surves: \url{http://edutechwiki.unige.ch/en/Usability_and_user_experience_surveys}

		\item Book: ``Evaluation Methods for Multimodal Systems: A Comparison of Standardized Usability Questionnaires'' (2008) - \url{http://link.springer.com/chapter/10.1007/978-3-540-69369-7_32}

		\end{enumerate}




	\textbf{4) Guidelines for the construction of Public Display Applications}

		\begin{enumerate}
		\item is this SECTION NEEDED?

		\item papers 23 (Overcoming Assumptions and Uncovering Practices, Huang), 25 (Worlds of Information, Jacucci): Best-practices / guidelines for designing good public display applications


		\item Further insights by Peltonen et al. \cite{peltonen2008s}: noticing others use the display, stepwise approach, parallel use, teamwork, conflict management, floor- and turn-taking, expressive and pondering gestures, and joint activities. The unsaid negotiation of social role taking between multiple users.
		\item Social behaviors arising from public display usage: crowding, massively parallel interaction, teamwork, games, negotiations of transitions and handovers, conflict management, gestures and overt remarks to co-present people, marking the display for others. \cite{peltonen2008s}


		\item \cite{redhead2009designing}: ``the display should clearly convey low commitment interaction, that it is quick and enjoyable. People preferred to interact with the display as objects, rather than via text based navigation.'' (find source, not sure whether I paraphrased it or if it's a citation)

		\end{enumerate}



	\textbf{4) Give an overview of other tools, possibly related to ours.}

		%\begin{itemize}[itemsep=0pt] 
		\begin{enumerate}
		\item give an overview of similar tools
		\item LimeSurvey (\url{http://de.wikipedia.org/wiki/LimeSurvey})
		\item SosciSurvey/LMU (\url{https://www.soscisurvey.de/})
		\item TODO look for more tools out there
		\item +Folgerungen aus anderen Bereichen (?)
		\end{enumerate}

		and clarify what the difference is between the already existing approaches to my approach.






% In the last part, restate that our approach is new, and that we are not aware of any similar approach before.


	\textbf{5a) Motivate why a platform such as the PDSurvey is of use. }

		\begin{itemize}[itemsep=0pt] 
		\item state how complex it is to administer / execute / conduct a survey or questionnaire
		\item encourage the motivation for creating a plattform like this!
		\item discrepancy between field and lab study: \cite{Ojala2011}.
		\item field studies are much more time consuming and usually spread over a larger area to assess. Being able to automate certain parts, such as the collection of quantitative or qualitative data, will facilitate the evaluation process for researchers and give new insights for display operators.
		\end{itemize}

	\textbf{5b) What is unique about our approach:}

		That we will have the opportunity to conduct surveys across a broad number of devices (large displays, tablets, smartphones, desktops), since they all access the same platform via a RESTful API. Which allows the greatest possible coverage of display providers' public displays and end consumer devices.

		We propose ...





