\section{Related Work}
\label{sec:related-work}

%\begin{itemize}[itemsep=0pt] 
	\begin{enumerate}
	\item give an overview of related work
	\item give background information to this thesis
	\item describe the work of others, what have they done so far?
	\end{enumerate}



%%% 
\subsection{Evaluation of Public Displays}

	Ovweview of papers with a relevance for the construction of surveys for public displays.


	\textbf{Creating questionnaires for public displays}
	\begin{enumerate}
	\item Jacucci: Worlds of Information (paper 25: \url{http://www.hiit.fi/u/morrison/chi2010.pdf})
	\end{enumerate}




%%%
\subsection{Conducting Surveys}

	\begin{itemize}[itemsep=0pt] 
	\item state how complex it is to administer / execute / conduct a survey or questionnaire
	\item encourage the motivation for creating a plattform like this!
	\item differentiate between surveys and questionnaires. survey = Umfrage, questionnaire = Fragebogen
	\item 
	\end{itemize}




%%% 
\subsection{Related Tools}

	Give an overview of other tools 

	%\begin{itemize}[itemsep=0pt] 
	\begin{enumerate}
	\item give an overview of similar tools
	\item LimeSurvey (\url{http://de.wikipedia.org/wiki/LimeSurvey})
	\item SosciSurvey/LMU (\url{https://www.soscisurvey.de/})
	\item TODO look for more tools out there
	\item +Folgerungen aus anderen Bereichen (?)
	\end{enumerate}

	and clarify what the difference is between the already existing approaches to my approach.





% In the last part, restate that our approach is new, and that we are not aware of any similar approach before. 
What is unique about our approach: That we will have the opportunity to conduct surveys across a broad number of devices (large displays, tablets, smartphones, desktops), since they all access the same platform via a RESTful API. Which allows the greatest possible coverage of display providers' public displays and end consumer devices.
