
% Table \ref{table:standardized-questionnaires} contains the following categories: \textbf{TODO TODO TODO}.
% Each one will be explained in more detail below.


	% User Experience  &&  Usability
	We distinguish between \textit{user experience} and \textit{usability} in our categorization, although it is hard and a controversial topic in literature \cite{bevan2009difference}.
	User experience describes the overall satisfaction and experience the user has with a display. Usability can be seen as a subcategory, however one difference is that usability can be measured based on hard facts such as response time, number of clicks, number of errors and has more to do with the effectiveness and efficiency. The evaluation of both user experience and usability can be carried out through questionnaires.

		>>> Sample questionnaires used for assessing the user 	experience are .... XXX TODO XXX


	% User Acceptance
		\textit{User acceptance} analyzes user's motives and incentives for approaching the display. The evaluation can be carried out qualitatively (subjective feedback, focus groups) or quantitatively (questionnaires).

	% Display Effectiveness
		Display effectiveness evaluates the economic perspective of display efficiency. 

	% Privacy
		Privacy takes a look at the users privacy concerns.

	% Social Impact
		Social impact considers everything related to social behavior, the influence on social interaction and communities, as well as social effects.

	% Context

		One new category is the collection of context data, relative to the public display. On most normal studies the context doesn't change during evaluation and thus is not as important. For the evaluation of public displays, especially when multiple displays are deployed in different locations running the same application, it will become if importance to also assess the static and dynamic context of each deployed display. External influences such as the weather, time of day, special events or semester break can have an influence on the number and type of people passing by a display. Additionally static context parameters, such as the display type, display size, position on wall, the size of the room might also influence how the display setup is perceived in public. Once recorded, these static and dynamic parameters can be evaluated with knowledge discovery algorithms for big data, a whole research field for itself. 
		So far no previous works are known on this area so far, evaluating a large public display deployment through an automated online platform with the help of context-based comparison. 

	
	% Demographics
		In most surveys \textit{demographic} background information about the participants is also of interest. This varies from general questions (gender, age, religion, education), more personal questions (relationship status, family, children, country of origin), skills (personal, professional, technical), personal beliefs, political affiliation or voluntary commitment.
		Three background questionnaires for inspiration, which we haven't used ourselves yet, but which go more in depth, are the Adult Literacy and Lifeskills Survey (ALL) \footnote{\url{http://nces.ed.gov/surveys/all/} (last accessed on April 1, 2015)}, the PIAAC Conceptual Framework of the Background Questionnaire Main Survey \footnote{\url{http://www.oecd.org/site/piaac/PIAAC(2011_11)MS_BQ_ConceptualFramework_1 Dec 2011.pdf} (last accessed on April 1, 2015)} and a Police Background Questionnaire \footnote{\url{http://www.slmpd.org/images/hr_forms/commissioned/BackgroundQuestionnaire.pdf} (last accessed on April 1, 2015)}.


	% Miscellaneous
		\textit{Miscellaneous} contains all of the questions and questionnaires, which can not be assigned to any of the previous categories 
		Cheverst et al. \cite{cheverst2005hermes} evaluated whether there were any previous experience with Bluetooth, or recommendations for possible new features. This can be 
% TODO REwRITE BOTH SENTENCES!
		For the evaluation of the Digifieds platform Alt et al. also evaluated: ``We asked them about their mobile phone usage (e.g., how often they used it, if it had a touch screen, if they used it to surf the web, and if they had installed third party apps) and whether they had used the UbiDisplays before'' \cite{alt2011digifieds}.


