\section{Literature Evaluation}
\label{chapter:literature-evaluation}

	Our research is based on an extensive literature review, with over 100 papers viewed. This led to the development of the public display survey platform (see chapter \ref{chapter:implementation}) and the categorization of standardized questionnaires (see section \ref{section:questionnaires:categorization}). A side effect of the literature review was, besides getting a better understanding of how public displays were evaluated, getting an overview of the questions asked to evaluate public displays and their applications. This turned out to be a quite valuable approach, since we haven't seen any compilation of questionnaires used for public display evaluation so far. Our goal was to find patterns and to build clusters of questionnaires being useful for the evaluation through automated public survey display platforms.

	% Overview of this chapter
	In the following we will first describe our methodology for gathering the information (section \ref{section:questionnaires:methodology}), followed by a categorization of standardized questionnaires (section \ref{section:questionnaires:categorization}), rounded off with the knowledge we draw from our review (section \ref{section:questionnaires:findings}). The categorization of the standardized questionnaires can be found in table \ref{table:standardized-questionnaires}.



\subsection{Methodology}
\label{section:questionnaires:methodology}
% My Process: Approach for Collecting Information

	The procedure for the selection of papers to review, was as follows. As a starting point all papers form the appendix of Florian Alt's doctoral thesis \cite{alt2013thesis} were read. Afterwards, interesting related work and citations were examined based on the papers from the previous step. This was supplemented with targeted research on Google Scholar and the APM Digital Library. To round off the literature review, publications of two authors who are very active in this field, were reviewed. 
	%A full list of all papers reviewed can be found in Appendix \ref{appendix:papers}.

	% 1) Appendix
	The first step was fairly straight forward. All papers of Florian Alt's appendix were read, in order to get a first overview of the current state of research. 
	% 2) Related Work / Citations
	The second step, pursuing related work and citations of interest, was carried out in a more subjective manner. Whenever interesting papers or projects were mentioned, the cited paper was skimmed through. 
	% 3) Google Scholar Search + ACM
	For the third step, a more strategic approach was used. Based on the insights gained from the previous steps, Google Scholar and APM was checked for literature relevant to our research question. The keywords used amongst others for the research in these online libraries were: \textit{standardized surveys for usability}, \textit{standardized surveys for user experience}, \textit{user satisfaction questionnaire}, \textit{public display evaluation}, and \textit{standardized public display evaluation}.
	% 4) Individual AUTHORS keywoard research
	The last step for collecting relevant papers consisted of profiling publications of two relevant authors in the area of public display research, namely J\"org M\"uller and Marcus Foth. The process started out by first finding a list of their publications. Since the literature review made by Florian Alt (see first step) already covered papers up to 2011, only ones published between 2012 and 2014 were viewed. 
	% procedure
	On each opened paper from this time frame a keyword search was carried out in order to see whether it contained an evaluation which might be relevant for us. These keywords were: \textit{questionnaire}, \textit{survey}, \textit{question}, \textit{interview}, \textit{(field) study}, and \textit{evaluation}. If none of these words could be found, the headlines and the abstract was skimmed through. All papers containing a reference to an evaluation of public displays were saved and analyzed in more detail.
	% authors
	For J\"org M\"uller the best list of his publications were found on his personal website\footnote{\url{http://joergmueller.info/publications.html} (last accessed on November 17, 2014)}, and for Marcus Foth two websites were evaluated \footnote{\url{http://www.vrolik.de/publications/} (last accessed on November 18, 2014) and \url{http://eprints.qut.edu.au/view/person/Foth,_Marcus.html} (last accessed on November 18, 2014)}. 

	% 5) Conference Proceedings
	% didn't have time for this step yet





\subsection{Standardized Questionnaires}
\label{section:questionnaires:categorization}

	% During the literature review phase a comprehensive list of widely used questionnaires was assessed. 

	As a result of the literature evaluation process the following overview of questionnaires arose. All questionnaires found during the literature review phase were categorized into a schema, inspired by the research questions introduced in chapter 2.8.2 of Florian Alt's doctoral thesis \cite{alt2013thesis}. Alt listed eight research categories: audience behavior, user experience, user acceptance, user performance, display effectiveness, privacy, and social impact. These categories served as a guideline for our classification of standardized questionnaires. However, since not all categories can be evaluated using questionnaires, we left out categories `audience behavior' and `user performance' for our summary. We extended the prior categorization with findings from the literature review phase and added four new categories: \textit{usability}, \textit{context}, \textit{demographics}, and \textit{other} for miscellaneous questions.
	A full list of standardized questionnaires found in literature can be found in table \ref{table:standardized-questionnaires}

	% For a list of papers using the standardized questionnaires for evaluation, refer to Appendix \ref{appendix:papers} or pages 54-55 from Alt \cite{alt2013thesis}. 

	% Other people's collections incorporated into our categorization can be found in the bibliography \cite{Lewis2013HCI, Garcia2013UXResearch, Geneve2014Wiki, Chur2014Questionnaires, wechsung2008evaluation}. 



	% % % % % % % % % % % % % % %
	%%%    ADD TABLE HERE    %%%%
	% % % % % % % % % % % % % % %

		\begin{table}[hpbt]
			\small
			\center
			%\begin{tabular}{p{3.5cm}p{5cm}p{6cm}c}
\toprule
\textbf{Category} & \textbf{Questionnaire} & \textbf{Description} & \textbf{Reference} \\ \midrule

\textbf{Usability} & SUS (System Usability Scale)  &    & \cite{alt2011digifieds}  \\
& USE  &  Usefulness, Satisfaction, and Ease of use  &   \\
& SUM  &  SUM provides one continuous variable for summative usability evaluations that can be used in regression analysis, hypothesis testing and usability reporting &  \\

& SUPR-Q  &  For measuring the perceptions of Usability, Trust, Credibility, Appearance and Loyalty of \textbf{Websites}  &   \\ \hline



\textbf{User Experience} & AttrakDiff Questionnaire  &  & \cite{Beyer2011}   \\ \hline
& UEQ / Short-UEQ  &  to evaluate user experience  &   \\


\textbf{User Acceptance} &  &  &    \\ \hline

\textbf{User Performance} & ISO 9241-9 based  &  & \cite{natapov2009iso}   \\ \hline



\textbf{Display Effectiveness} &  &  &    \\ \hline

\textbf{Privacy} &  &  &    \\ \hline

\textbf{Social Impact} &  &  &    \\ \hline

\textbf{Context} &  & Determine in which context the interaction with the public display took place &    \\ \hline

\textbf{Demography} &  & Getting to know the user (Background Questionnaire, User Information) &    \\

& PIAAC & &   \\
& AAL & Adult Literacy and Lifeskills Survey & \url{http://nces.ed.gov/surveys/all/}  \\




\textbf{Other} &  & Whatever does not fit into any of the previous categories (comments, general feedback) &    \\ \hline

& GameFlow & Measures the Flow & Paper 22  \\
& MEC & Presence & Paper 22   \\
& IMI & Intrinsic Motivation & Paper 22  \\
& Nielsen's system acceptance model & & \cite{ojala2010ubi}  \\



\textit{not sure where to put them} &  & Motivation, Motivating Factors, Expectations, User Goals, Collaboration, Awareness
 &    \\ \hline


\textbf{not sure} & QUIS  & Questionnaire for User Interaction Satisfaction  &   \\
& ISO9241-9 based  & Questionnaire for User Interaction Satisfaction  &   \\


\bottomrule
\end{tabular}

			\caption[Standardized Questionnaires]{Overview of Standardized Questionnaires}
			\label{table:standardized-questionnaires}
		\end{table}

	% % % % % % % % % % % % % % %
	%%%    ADD TABLE HERE    %%%%
	% % % % % % % % % % % % % % %
	
	% > ADD A TABLE with all of the questionnaires, a short description, the date, the number of questions, and the reference to the bib file.



	We distinguish between \textit{user experience} and \textit{usability} in our categorization, although it is a controversial topic in literature \cite{bevan2009difference}.
	User experience describes the overall satisfaction and experience the user has with a display. Usability can be seen as a subcategory, however one difference is that usability can be measured based on hard facts such as response time, number of clicks, number of errors and has more to do with the effectiveness and efficiency. The evaluation of both user experience and usability can be carried out through questionnaires. %+ Questionnaires which are suitable for this purpose, ...

	\textit{User acceptance} analyzes user's motives and incentives for approaching the display. The evaluation can be carried out qualitatively (subjective feedback, focus groups) or quantitatively (questionnaires).

	\textit{Display effectiveness} evaluates the economic perspective of display efficiency. 

	\textit{Privacy} takes a look at the users privacy concerns.

	The category \textit{social impact} considers everything related to social behavior, the influence on social interaction and communities, as well as social effects.



	% CONTEXT
	One newly added category is \textit{context}. For most evaluations of computer applications, the context changes infrequently. However, for the evaluation of public displays, especially when multiple displays are deployed in different locations running the same application, it is important to also assess the context of the display setup. External influences can be both of static and dynamic nature. Influences such as weather, time of day, or special circumstances in the displays' environment all count as \textit{dynamic context}. To parameters like display size, display type, position on wall, position in room, or size of the room count are referred to as \textit{static context}. These characteristics can vary from display setup to display setup and can influence how the public display application is perceived.
	So far, no previous works are known on this area, evaluating a public display deployments with the help of context-based comparison. 

	% Once logged, these static and dynamic parameters can be evaluated with knowledge discovery algorithms for big data, a whole research field for itself. 

	% DEMOGRAPHICS
	In most surveys \textit{demographic} background information are evaluated. This questions vary from general (gender, age, religion, education), to more personal questions (relationship status, family, children, country of origin). Often times also character traits, skills, personal beliefs, or political affiliation are of interest.
	Three detailed background questionnaires which we haven't used ourselves yet, but might serve for inspiration, are the Adult Literacy and Lifeskills Survey (ALL) \footnote{\url{http://nces.ed.gov/surveys/all/} (last accessed on April 1, 2015)}, the PIAAC Conceptual Framework of the Background Questionnaire Main Survey \footnote{\url{http://www.oecd.org/site/piaac/PIAAC(2011_11)MS_BQ_ConceptualFramework_1 Dec 2011.pdf} (last accessed on April 1, 2015)} and a Police Background Questionnaire \footnote{\url{http://www.slmpd.org/images/hr_forms/commissioned/BackgroundQuestionnaire.pdf} (last accessed on April 1, 2015)}.

	% Miscellaneous
	\textit{Miscellaneous} contains all of the questions and questionnaires, which can not be assigned to any of the previous categories. As an example, Cheverst et al. \cite{cheverst2005hermes} asked for recommendations for possible new features and whether there were any previous experience with Bluetooth. Alt et al. \cite{alt2011digifieds} evaluated more detailed usage patterns regarding mobile phone usage, ``e.g., how often they used it, if it had a touch screen, if they used it to surf the web, and if they had installed third party apps''. These types of questions do not directly belong to demographic survey, but are part of many questionnaires.



	
% Table \ref{table:standardized-questionnaires} contains the following categories: \textbf{TODO TODO TODO}.
% Each one will be explained in more detail below.


	% User Experience  &&  Usability
	We distinguish between \textit{user experience} and \textit{usability} in our categorization, although it is hard and a controversial topic in literature \cite{bevan2009difference}.
	User experience describes the overall satisfaction and experience the user has with a display. Usability can be seen as a subcategory, however one difference is that usability can be measured based on hard facts such as response time, number of clicks, number of errors and has more to do with the effectiveness and efficiency. The evaluation of both user experience and usability can be carried out through questionnaires.

		>>> Sample questionnaires used for assessing the user 	experience are .... XXX TODO XXX


	% User Acceptance
		\textit{User acceptance} analyzes user's motives and incentives for approaching the display. The evaluation can be carried out qualitatively (subjective feedback, focus groups) or quantitatively (questionnaires).

	% Display Effectiveness
		Display effectiveness evaluates the economic perspective of display efficiency. 

	% Privacy
		Privacy takes a look at the users privacy concerns.

	% Social Impact
		Social impact considers everything related to social behavior, the influence on social interaction and communities, as well as social effects.

	% Context

		One new category is the collection of context data, relative to the public display. On most normal studies the context doesn't change during evaluation and thus is not as important. For the evaluation of public displays, especially when multiple displays are deployed in different locations running the same application, it will become if importance to also assess the static and dynamic context of each deployed display. External influences such as the weather, time of day, special events or semester break can have an influence on the number and type of people passing by a display. Additionally static context parameters, such as the display type, display size, position on wall, the size of the room might also influence how the display setup is perceived in public. Once recorded, these static and dynamic parameters can be evaluated with knowledge discovery algorithms for big data, a whole research field for itself. 
		So far no previous works are known on this area so far, evaluating a large public display deployment through an automated online platform with the help of context-based comparison. 

	
	% Demographics
		In most surveys \textit{demographic} background information about the participants is also of interest. This varies from general questions (gender, age, religion, education), more personal questions (relationship status, family, children, country of origin), skills (personal, professional, technical), personal beliefs, political affiliation or voluntary commitment.
		Three background questionnaires for inspiration, which we haven't used ourselves yet, but which go more in depth, are the Adult Literacy and Lifeskills Survey (ALL) \footnote{\url{http://nces.ed.gov/surveys/all/} (last accessed on April 1, 2015)}, the PIAAC Conceptual Framework of the Background Questionnaire Main Survey \footnote{\url{http://www.oecd.org/site/piaac/PIAAC(2011_11)MS_BQ_ConceptualFramework_1 Dec 2011.pdf} (last accessed on April 1, 2015)} and a Police Background Questionnaire \footnote{\url{http://www.slmpd.org/images/hr_forms/commissioned/BackgroundQuestionnaire.pdf} (last accessed on April 1, 2015)}.


	% Miscellaneous
		\textit{Miscellaneous} contains all of the questions and questionnaires, which can not be assigned to any of the previous categories 
		Cheverst et al. \cite{cheverst2005hermes} evaluated whether there were any previous experience with Bluetooth, or recommendations for possible new features. This can be 
% TODO REwRITE BOTH SENTENCES!
		For the evaluation of the Digifieds platform Alt et al. also evaluated: ``We asked them about their mobile phone usage (e.g., how often they used it, if it had a touch screen, if they used it to surf the web, and if they had installed third party apps) and whether they had used the UbiDisplays before'' \cite{alt2011digifieds}.






	% Other people's collections
	A list of other people's collections of standardized questionnaires can be found in the bibliography. Lewis and Sauro \cite{Lewis2013HCI} list 19 questionnaires at the HCI conference. Garcia \cite{Garcia2013UXResearch} describes the SUMI, PSSUQ, and SUS questionnaire. The Universit{\'e} de Gen{\`e}ve \cite{Geneve2014Wiki} gives an overview of usability and user experience surveys. HTW Chur \cite{Chur2014Questionnaires} provides an overview of ISONorm 110, ISOMetrics, AttrakDiff, UEQ, QUIS, and SUMI. For further information regarding standardized usability questionnaires and evaluation methods for multimodal systems the book by Wechsung and Naumann \cite{wechsung2008evaluation} can be used. 





\subsection{Findings}
\label{section:questionnaires:findings}

	The following results were drawn from the literature review, which can be turn into requirements for our or future survey platforms (see chapter \ref{chapter:implementation}).


	\begin{enumerate}
		\item Support both \textit{quantitative} and \textit{qualitative} methods for data collection. Since not all research questions can be evaluated only with questionnaires. In the long term it will be necessary to support logging, observations, and/or interviews.

		\item Support \textit{multiple} sections. Many questionnaires are built up of different sections. These can either be displayed all at once, or spread across multiple pages, and maybe also be spread across multiple users (see chapter \ref{chapter:future-work}, future work).

		\item Support various \textit{question types}. The analyzed questionnaires use a variety of question types for evaluation. Amongst others, 5-point Likert scale, 7-point Likert scale, multiple choice responses, numeric answers, text fields, yes-no questions.

		\item Evaluate not only the application running on the display, but also the entire environment. Differences in the context of the public display often result in different perceptions and user interaction.

		\item One constraint of public display research represents the opportunistic nature of the setups and the discrepancy between lab studies and field studies \cite{Ojala2011}. Thus there is an additional demand for evaluating each public display setup individually and if possible directly in the field.

		\item Additionally a larger number of form factors, platforms and end devices needs to be supported, to cover the whole range of public displays being out there
	\end{enumerate}


	% Transfer
	These findings from literature review bring us to the next chapter, the development of the survey platform.

