\section{Questionnaires}
\label{sec:questionnaires}

As a result of the literature review, besides getting a better understanding of how public displays were evaluated so far, a side effect was getting an overview of the questions asked to evaluate public displays and their applications. This turned out to be a quite valuable approach, since we haven't seen any compilation of  questionnaires used for public display evaluation so far. Our goal was to find patterns and to build clusters of questionnaires being useful for the evaluation through automated public survey display platforms.

% ALTERNATIVES FOR FIRST SENTENCE

	% Through the literature review, it was a side effect to get a better overview of the questions asked to evaluate public displays and of their applications running on them.

	% As a side effect of the literature review,  a side effect was getting an overview of the questions asked to evaluate public displays and of the applications running on them.

	% While performing the literature review, not only getting a better understanding of how public displays are evaluated was a result, but also getting a better overview of the questions asked to evaluate public displays and of their applications running on them


In the following we will ...

	% + overview of this chapter

TODO: think about the common theme, maybe rename this chapter



\subsection{Approach for Collecting Information}
% My Process

	The procedure for the selection of papers to review, was as follows. As a starting point all papers, which are in the appendix of Florian Alt's doctoral thesis\cite{alt2013thesis} were read, to get a first overview. Afterwards interesting related work and citations were followed based on the papers from the previous step. This was supplemented with targeted research on Google Scholar based on the following keywords: \textbf{TODO TODO TODO}. A full list of all papers reviewed can be found in Appendix, section \ref{appendix:papers}.

	% 1) Appendix
		% and related work from these papers, Google Scholar, ACM
	The analysis of the appendix was fairly straight forward. All papers were read from start to finish (pages 335 to 343). 

	% 2) Related Work / Citations
	The second step, pursuing related work and citations of interest, was carried out in a more subjective manner.

	% 3) Google Scholar Search + ACM
		% used Keywoard approach (see Wiki)

		searching for ``Standardized Surveys for Usability / User Experience''



	% 4) Individual AUTHORS
	The last step for collecting relevant papers consisted of profiling publications of two relevant authors in the area of public display research, namely J\"org M\"uller and Marcus Foth. The process started out by finding a list of their publications. Since the literature review made by Florian Alt (see first step) already covered papers up to 2011, only ones published between 2012 and 2014 were viewed. 
	% procedure
	On each opened papers from this time frame a keyword search was carried out, to see whether it contained an evaluation which might be relevant for us. These keywords were: \textit{questionnaire}, \textit{survey}, \textit{question}, \textit{interview}, \textit{(field) study}, and \textit{evaluation}. If none of these words could be found, the headlines and the abstract was skimmed through. All papers containing a reference to evaluating public displays were saved locally and analyzed in more detail.
	% authors
	For J\"org M\"uller the best list of his publications were found on his personal website\footnote{\url{http://joergmueller.info/publications.html} (accessed on November 17, 2014)}, and for Marcus Foth two websites were evaluated \footnote{\url{http://www.vrolik.de/publications/} (accessed on November 18, 2014) and \url{http://eprints.qut.edu.au/view/person/Foth,_Marcus.html} (accessed on November 18, 2014)}. 

	% 5) Conference Proceedings
		% didn't have time for this step yet, should I just not mention it?





\subsection{Categorization}

	As a result of the literature review process the following overview of questionnaires arose. 

	Categorization of all the question and questionnaires found in literature, during the review process.

	\paragraph{Clustering of question types}

	\paragraph{Constructing a unified/standardized Questionnaire}

	\paragraph{5-point or 7-point Likert scale}
	http://www.measuringu.com/blog/scale-points.php




		%%% ADD TABLE %%%
		% http://www.tablesgenerator.com/
		% http://truben.no/table/

		\label{table:ASDFTODO}

%		\begin{table*}[htbp]
%		  \centering
%		  \caption{TODO TABLE DESCRIPTION / CAPTION}

\begin{table}[h]
\begin{tabular}{|p{4cm}|p{6cm}|p{3cm}|p{2cm}|}
\textbf{Categories}                   & \textbf{Description}                                                  & \textbf{Types of Questions / Questionnaires} & \textbf{Papers / Links} \\ \hline
Usability                             &                                                                       &                                              &                         \\ \hline
User Experience                       &                                                                       &                                              &                         \\ \hline
Awareness                             &                                                                       &                                              &                         \\ \hline
Social Aspects \& Collaboration       &                                                                       &                                              &                         \\ \hline
Expectations \& User Goals            &                                                                       &                                              &                         \\ \hline
Motivating Factors                    &                                                                       &                                              &                         \\ \hline
Motivation (reactions to the display) &                                                                       &                                              &                         \\ \hline
User Engagement                       &                                                                       &                                              &                         \\ \hline
Privacy                               &                                                                       &                                              &                         \\ \hline
Demographics                          & User Information, Background Questionnaire, Getting to know the User  &                                              &                         \\ \hline
Multi-User Interaction                & Parallel Use                                                          &                                              &                         \\ \hline
(Information Sharing)                 &                                                                       &                                              &                         \\ \hline
Interaction Context                   & Let the user assess in which context the survey took place            &                                              &                         \\ \hline
Comments                              & Leaving room for general feedback                                     &                                              &                         \\ \hline
Misc                                  & Whatever I forgot or doesn't fit into any of the previous categories. &                                              &                         \\
                                      &                                                                       &                                              &                         \\
                                      &                                                                       &                                              &                         \\
                                      &                                                                       &                                              &                         \\
                                      &                                                                       &                                              &                        
\end{tabular}
		\begin{tablenotes}
		      \small
		      \item Categorization ... table notes 
		\end{tablenotes}
		\caption[Short Caption]{And here is the long caption for this table, going more in detail.}
\end{table}		    
		%\begin{tablenotes}
		%      \small
		%      \item TODO TABLE NOTES 
		%\end{tablenotes}

	%	\end{table*}

		%%% END OF TABLE %%%




\subsection{Standardized Questionnaires}

	% List all the standardized questionnaires I found during my research phase.

		During the literature review phase a comprehensive list of widely used questionnaires was assessed. A first overview of other people's summaries can be found here

		\begin{enumerate}
		\item \url{http://2013.hci.international/index.php?module=pagesmith&uop=view_page&id=44}
		\item \url{http://edutechwiki.unige.ch/en/Usability_and_user_experience_surveys}
		\item \url{http://chaione.com/ux-research-standardizing-usability-questionnaires/}
		\item \url{http://www.cheval-lab.ch/was-ist-usability/usabilitymethoden/frageboegen/}
		\item \url{https://docs.google.com/document/d/1D925jJ7bmRc1EZdCTz32lmW2hniMiq7GzBWxX8rmhpE/edit}
		\end{enumerate}

	And later state which ones I chose to use and why.

		In the following a full overview of all standardized questionnaires found in literature can be found, grouped by superordinate categories.

		BRING A TABLE WITH ALL OF THE QUESTIONNAIRES, A SHORT DESCRIPTION, THE DATE, THE NUMBER OF QUESTIONS, THE REFERENCE HERE


\subsection{Findings}

	Findings: use both quantitative and qualitative methods for data collection (explain why this is important, teaser this as a requirement for the platform, how it could be implemented)



%%% Transfer to the next chapter

In the following chapter we will talk about ...