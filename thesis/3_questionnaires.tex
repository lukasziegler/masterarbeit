\section{Standardized Questionnaires}
\label{sec:questionnaires}
% TODO: think about the common theme, maybe rename this chapter

As a result of the literature review, besides getting a better understanding of how public displays were evaluated so far, a side effect was getting an overview of the questions asked to evaluate public displays and their applications. This turned out to be a quite valuable approach, since we haven't seen any compilation of questionnaires used for public display evaluation so far. Our goal was to find patterns and to build clusters of questionnaires being useful for the evaluation through automated public survey display platforms.

% Overview of this chapter
In the following we will first describe how all information was gathered, how the data was evaluated and condensed to the schema shown in Table \ref{table:standardized-questionnaires}.





\subsection{Approach for Collecting Information}
% My Process

	The procedure for the selection of papers to review, was as follows. As a starting point all papers form the appendix of Florian Alt's doctoral thesis \cite{alt2013thesis} were read. Afterwards interesting related work and citations were followed based on the papers from the previous step. This was supplemented with targeted research on Google Scholar and the APM Digital Library. To round off the literature review the publications of two authors, who are very active in this field, were reviewed. A full list of all papers reviewed can be found in Appendix \ref{appendix:papers}.

	% 1) Appendix
	The first step, the analysis of the appendix was fairly straight forward. All papers were read from start to finish (pages 335 to 343), in order to get a first overview of the current state of research. 
	% 2) Related Work / Citations
	The second step, pursuing related work and citations of interest, was carried out in a more subjective manner. Whenever interesting papers or projects were mentioned, the cited paper was also skimmed through. 
	% 3) Google Scholar Search + ACM
	For the third step, a more strategic approach was used. Based on the insights gained from the previous steps, Google Scholar and APM was checked for literature relevant to our research question. The Keywords, that were used amongst others for the research in these online libraries, were:
	\begin{itemize}[itemsep=0pt] 
	\item Standardized Surveys for Usability
	\item Standardized Surveys for User Experience
	\item user satisfaction questionnaire
	\item public display evaluation
	\item standardized public display evaluation
	\end{itemize}

	% 4) Individual AUTHORS keywoard research
	The last step for collecting relevant papers consisted of profiling publications of two relevant authors in the area of public display research, namely J\"org M\"uller and Marcus Foth. The process started out by firstM finding a list of their publications. Since the literature review made by Florian Alt (see first step) already covered papers up to 2011, only ones published between 2012 and 2014 were viewed. 
	% procedure
	On each opened papers from this time frame a keyword search was carried out, to see whether it contained an evaluation which might be relevant for us. These keywords were: \textit{questionnaire}, \textit{survey}, \textit{question}, \textit{interview}, \textit{(field) study}, and \textit{evaluation}. If none of these words could be found, the headlines and the abstract was skimmed through. All papers containing a reference to evaluating public displays were saved locally and analyzed in more detail.
	% authors
	For J\"org M\"uller the best list of his publications were found on his personal website\footnote{\url{http://joergmueller.info/publications.html} (accessed on November 17, 2014)}, and for Marcus Foth two websites were evaluated \footnote{\url{http://www.vrolik.de/publications/} (accessed on November 18, 2014) and \url{http://eprints.qut.edu.au/view/person/Foth,_Marcus.html} (accessed on November 18, 2014)}. 

	% 5) Conference Proceedings
		% didn't have time for this step yet, should I just not mention it?





\subsection{Categorization of Questionnaires}


	% As a result of the literature review process the following overview of questionnaires arose. 
		% % OR % %
	% The findings from literature review were fitted to the categories presented in  research questions found in 

	Based on Alt's \cite{alt2013thesis} findings the following research questions were popular in the past, serving as a guidline for our classification of standardized questionnaires: audience behaviour, user experience, user acceptance, user performance, display effectiveness, privacy, and social impact. 

		% \textit{Audience behavior} includes effects such as the honeypot, sweet spot and landing zone, which can be assessed ``observations and log data''. The focus lies on ``how the audience behaves around a display''.

		\textit{User experience} describes the overall satisfaction and experience the user has with a display. The evaluation can be carried out through questionnaires.
		\textit{User acceptance} analyzes user's motives and incentives for approaching the display. The evaluation can be carried out qualitatively (subjective feedback, focus groups) or quantitatively (questionnaires).
		% \textit{User performance} measures the effectiveness from a user's perspective, based on quantitative measures such as task completion time or error rates.
		\textit{Display effectiveness} evaluates the economic perspective of display efficiency. 
		\textit{Privacy} takes a look at the users privacy concerns.
		\textit{Social impact} considers everything related to social behavior, the influence on social interaction and communities, as well as social effects.

	For citations and practical examples for the above mentioned research questions, refer to \cite{alt2013thesis}, pages 54 to 55.




	In the following a full overview of all standardized questionnaires found in literature can be found, grouped by superordinate categories.


		% % % % % % % % % % % % % % %
		%%%    ADD TABLE HERE    %%%%
		% % % % % % % % % % % % % % %

		\label{table:standardized-questionnaires}

			% ADD A TABLE WITH ALL OF THE QUESTIONNAIRES, A SHORT DESCRIPTION, THE DATE, THE NUMBER OF QUESTIONS, THE REFERENCE HERE

		\begin{enumerate}
			\item POST OR REFERENCE THE TABLE WITH THE CATEGORIZATION HERE
			\item My Categorization: \url{https://docs.google.com/document/d/1D925jJ7bmRc1EZdCTz32lmW2hniMiq7GzBWxX8rmhpE/edit} (Google Docs)
		\end{enumerate}

		% % % % % % % % % % % % % % %
		%%%    ADD TABLE HERE    %%%%
		% % % % % % % % % % % % % % %





\subsection{Findings}

	Findings: 
	\begin{enumerate}
	\item use both quantitative and qualitative methods for data collection (explain why this is important, teaser this as a requirement for the platform, how it could be implemented)
	\item support mutliple sections, all displayed at once or (optionally) spread across multiple users
	\item support various question types (e.g. 5-point and 7-point Likert scale, multiple choice, numeric responses, comments)
	\end{enumerate}




%%% Transfer to the next chapter

	These findings bring us to the next chapter, the research platform to develop, capable of conducting all of these questionnaires.

	% In the following chapter we will talk about ...












% ALTERNATIVES FOR FIRST SENTENCE

	% Through the literature review, it was a side effect to get a better overview of the questions asked to evaluate public displays and of their applications running on them.

	% As a side effect of the literature review, a side effect was getting an overview of the questions asked to evaluate public displays and of the applications running on them.

	% While performing the literature review, not only getting a better understanding of how public displays are evaluated was a result, but also getting a better overview of the questions asked to evaluate public displays and of their applications running on them

