\section{Implementation}
\label{sec:implementation}
	% Chapter concerning the Technical Realization

	In this chapter we will deal with the infrastructure and technical realization of the public display survey platform. First off we will start with the requirements for the survey platform (section \ref{sec:implementation:requirements}). Subsequently the architecture resulting from the design decisions will be our main focus (section \ref{sec:implementation:design-decisions}). To facilitate the training period for successors we will also take a brief look at the software model (section \ref{sec:implementation:modeling}). For more specific information and for information regarding maintenance of the project, please refer to the Documentation found on the CD enclosed or on the GitHub repository (see Appendix \ref{appendix:documentation}).

	In figure \ref{fig:4-pdsurvey-platform} a brief overview of the \textit{PDSurvey} platform and its components is given. The platform consists of three major parts: a backend for display providers (PDAdmin), a RESTful server (PDServer) and the user interface itself, being embedded on the end user devices (public displays, tablets, smartphones or other devices). 






\subsection{Requirements}
\label{sec:implementation:requirements}

	The starting point for the PDSurvey platform was the Master' thesis announcement, which already included first requirements (see page \pageref{aufgabenstellung}). 

	% Official Problem statement
	\begin{enumerate}[itemsep=0pt] 
	\item development of a survey tool that allows interactive public display installations to be comprehensively assessed 
	\item a web-based survey platform will be implemented that can easily be used to evaluate and compare public displays through different channels 
	\item different channels to support: 1) evaluation directly at
	the display or 2) through a (mobile) website that allows participation also via a smartphone
	or tablet.
	\item configuration options for public display owners
	\end{enumerate}

	After the analysis of the thesis' problem statement, the following requirements were derived, already having an impact on the later chosen architecture:

	% Derived requirements
	\begin{enumerate}[itemsep=0pt] 
	\item JavaScript embed code, for public displays
	\item supporting multiple devices: public displays, mobile (smartphone, tablet), desktop
	\item responsive web design
	\end{enumerate}

	Additional requirements, self proposed:
	\begin{enumerate}[itemsep=0pt] 
	\item scalable
	\item modular / extensible
	\item multilingual / internationalization (i18n)
	\end{enumerate}






\clearpage
% 


%%  DATABASE  %%
\subsubsection{Database}


\paragraph{SQL: MySQL}

\begin{enumerate}
\item http://www.mysql.com/
\item "MySQL ist aus gutem Grund das Arbeitstier der Open-Source-Welt geworden: Es bietet viele der M\"oglichkeiten grosser kommerzieller Datenbanken - und ist dabei kostenlos. In seiner aktuelen Form ist MySQL performant und besitzt viele Features." \cite{hughes2012einfuhrung} (Quelle: Buch - Einf\"uhrung in Node.js, Tom Hughes-Croucher \& Mike Wilson, O'Reilly, Seite 134)
\end{enumerate}


\paragraph{SQL: PostgreSQL}

\begin{enumerate}
\item http://www.postgresql.org/
\item "Postgres (sp\"ater PostgreSQL) ist ein objektorientiertes RDBMS, das urspr\"unglich an der University of California in Berkeley entwickelt wurde. Vater des Projekts war Professor Michael Sonebraker, der es als Nachfolger seines \"alteren Ingres-Datenbank-Systems ansah. [...] Nach der Zeit in Berkeley übernahmen Open-Source-Entwickler das Projekt, ersetzten den ursprünglichen QUEL-Sprachinterpreter durch einen SQL-Sprachinterpreter und benannen das Projekt in PostgreSQL um." \cite{hughes2012einfuhrung} (Quelle: Buch - Einf\"uhrung in Node.js, Tom Hughes-Croucher \& Mike Wilson, O'Reilly, Seite 141)
\end{enumerate}


\paragraph{NoSQL: CouchDB}

\begin{enumerate}
\item http://couchdb.apache.org/
\item "CouchDB bietet einen MVCC-basierten (Multi-Version Concurrency Control) Dokumentenspeicher in einer JavaScript Umgebung. Werden Dokumente (Datens\"atze) in CouchDB hinzugef\"ugt oder aktualisiert, dann wird der gesamte Datensatz gespeichert und \"altere Versionen der Daten als obsolet markiert. Diese \"alteren Versionen des Datensazes k\"onnen immer noch in die neueste Version \"ubernommen werden, aber es wird immer eine neue Version erzeugt und f\"ur einen schnellen Lesezugriff in fortlaufenden Speicherbereichen abgelegt" \cite{hughes2012einfuhrung} (Quelle: Buch - Einf\"uhrung in Node.js, Tom Hughes-Croucher \& Mike Wilson, O'Reilly, Seite 113)
\end{enumerate}

\paragraph{NoSQL: Redis}

\begin{enumerate}
\item http://redis.io/
\item "Redis ist eine In-Memory-Datenbank mit einer Schl\"ussel/Wert-Ablage, die Ihnen sehr vertraut vorkommen sollte, wenn Sie schon \"uber Erfahrungen mit Schl\"usse/Wert-Caches wie Memcache verf\"ugen. Redis wird verwendet, wenn Performance und Skalierbarkeit wichtig sind. In vielen F\"allen entscheiden sich die Entwickler daf\"ur, es als Cache f\"ur die von einer relationealen Datenbank wie MySQL empfangenen Daten einzusetzen, auch wenn es deutlich mehr kann." \cite{hughes2012einfuhrung} (Quelle: Buch - Einf\"uhrung in Node.js, Tom Hughes-Croucher \& Mike Wilson, O'Reilly, Seite 121)
\end{enumerate}

\paragraph{NoSQL: MongoDB}

"NoSQL databases came back into the mainstream when developers needed better performance and were ok with giving up the relational aspect of RDBs (unions, joins, etc)" \url{http://psitsmike.com/2012/02/node-js-and-mongo-using-mongoose-tutorial/}

\begin{enumerate}
\item http://www.mongodb.org/
\item MongoDB is document database, storing all data in the form of JavaScript objects, PHP arrays, Python dicts or Ruby hashes. This significantly facilitates the handover from JavaScript to the database management system (DBMS).
\item "Da Mongo eine JavaScript-Umgebung mit BSON-Objektspeichern (einer Bin\"ar-Adaption von JSON) mitbringt, ist das Lesen und Schreiben von Daten aus Node heraus ausgesprochen effizient. Mongo speichert eintreffende Datens\"atze im Speicher, daher ist es in Situationen mit viel Schreibvorg\"angen eine ideale Wahl. Jede neue Version bietet verbessertes Clustering, bessere Replikation und Sharding. Da die eintreffenden Datens\"atze im Speicher abgelegt werden, ist das Einf\"ugen von Daten in Mongo nicht blockierend, womit es ideal f\"ur das Protokollieren von Operationen und Telemetriedaten ist. Mongo unterst\"utzt JavaScript-Funktionen in Abfragen. Dadurch ist es beim Lesen sehr leistungsf\"ahig, so zum Beispiel bei MapReduce-Anfragen. Mit dem dokumentenbasierten Datenspeicher von MongoDB k\"onnen Sie in Eltern-Datens\"atzen auch Kind-Datens\"atze ablegen. So kann zum Beispiel ein Blogartikel mit all seinen Kommentaren in einem einzelnen Datensatz gespeichert werden, wodurch er sich auch wieder sehr schnell auslesen l\"asst."  \cite{hughes2012einfuhrung}
\end{enumerate}

Reasons for using MongoDB
\begin{enumerate}
\item ideal for lots of write procedures
\item non-blocking write operations, ideal for Node.js and for logging data (good for our future work)
\item good read performance
\item good scalability
\item fully supports JSON syntax
\item good integration with Node.js, see Mongoose\footnote{http://mongoosejs.com/}
\end{enumerate}


Notes from the MondoDB University M101JS Class
\begin{enumerate}
\item is non-relational, ideal for JSON data
\item MongoDB is schemaless. Two documents in the same collection can have different schemas.
\item MongoDB provides a good compromise between scalability/performance and the depth of funcitonality
\item Drawback: MongoDB does not support Joins or Transactions
\end{enumerate}


\subsection{Design Decisions}
\label{sec:implementation:design-decisions}

	% Due to the requirements discussed in section ...

	After having assessed all requirements for the platform (see \ref{sec:implementation:requirements}), the next step was making design decisions for the software, programming language and frameworks to use, having an impact on the architecture.



	\begin{enumerate}
	\item \textbf{Programing language}
	\item due to the the requirements and objective to support a large number of devices, JavaScript is the choice


	\item \textbf{Frontend Framework}
	\item AngularJS + Bootstrap
	\item link to 2 articles with comparisions
	\item say why I chose these two: best documentation, big community (fast support), allowing me for making fast progress


	\item \textbf{Backend}
	\item give a short overview of alternatives
	\item say because of embedding + frontend, it seems obvious to choose JavaScript
	\item Server: NodeJS is currently becoming the de-facto standard for scalable and RESTful web services.


	\item \textbf{Interaction / Communication}
	
	\item \textbf{Database}
	\item SQL vs NoSQL: NoSQL better for 
	\item MongoDB is the most popular NoSQL database as of now and integrates seamlessly with the MEAN stack (MongoDB, ExpressJS, AngularJS, NodeJS).

	\item \textbf{Hosting}
	\item copy + paste!

	\end{enumerate}



Bits and pieces

		Should a component not support HTML and JavaScript execution, then the required surveys can still be communicated directly with the REST API of PDServer.

		For administrative purposes we created a RESTful backend (PDBackend), which enables the creation, management and distribution of surveys to public displays. Attached to this backend there is the   a responsive client 









\clearpage
\subsection{Modeling}
\label{4c_modeling}

The model for the \textit{PDSurvey} platform is primarily defined in Mongoose schema, which serves as the connection to the database. Node.js maps the route parameters and routes all requests to the corresponding Mongoose model. Angular.js builds its model upon the REST API and maps it via dynamic two-way-binding to it's scope. Thus all changes to the model originate from Mongoose.


\subsubsection{Software model}

In total there are the following classes.

\begin{enumerate}
\item list all classes
\item 1 UML diagram is enough, according to Florian
\end{enumerate}


\subsubsection{Dependencies}

Of special interest are the following four models: Survey, Display, Campaign and Responses.

\textbf{Surveys} resembles the foundation of PDSurvey, with the aim of reuse and standardization. A survey consists of multiple sections, being built of of multiple questions. Each question is of a corresponding question type and every survey belongs to a category. This allows the filtering for relevant surveys. To be able to create private surveys, not being shared across the entire platform, every survey is assigned to an individual user.

In the \textbf{display} collection all displays connected to the PDSurvey platform are contained. To allow for an evaluation across multiple display models and based on the context of the displays, the display model and a static and/or dynamic context is assigned to every display.

\textbf{Campaigns} resemble the most integral part, since they glue all of the pieces together and allow the distribution of surveys to public display networks. A campaign consists of displays and surveys and creates the mapping of the questionnaires to public displays. Additionally to each of those mapping an individual context can be assigned, enabling the later comparison of results in between the public displays.

All \textbf{responses} made to each survey are logged in the Response collection. The queries are carried out individually per user, per display and per campaign. This model will be the base for further extensions, amongst others the automatic evaluation of the survey responses and the comparison inbetween an entire display network, to be able to find out which properties of a display might be related to certain effects.


\begin{figure}%[btph]
    \begin{center}
        \includegraphics[width=.8\columnwidth]{img/4_implementation/4-dependency-campaign}
    \end{center}
 % \begin{center}\LARGE [BILD]\end{center}
 \caption{Campaign model dependencies}
 \label{fig:4-dependency-campaign}
\end{figure}


++ neue Namensgebung, um in der domain specific language zu bleiben --> application provider / display provider / space provider (anstatt Operator). Wir werden aber nur mit dem Application Provider (anstatt Operator) im System arbeiten




\subsubsection{User stories}

Concept from extreme programming.

\begin{enumerate}
\item inspired by: ...
\item criteria for good user stories: http://tigertechtalk.wordpress.com/2012/10/17/wie-schreibe-ich-eine-gute-user-story-und-was-ist-das-uberhaupt/
\end{enumerate}



+ User Centered Design (paper nr 31): ``constitutes an iterative process of system design, deployment and evaluation'' (quote from paper 31)






\subsubsection{User roles}

As of now only two roles are implemented, the admin-role and the guest-role.

% FEEDBACK VON FLORIAN: nein, nicht zu komplex machen. Es sollte sogar reichen, nur zwischen Admin & Operator (= Application Provider) zu unterscheiden, da der end user (das Public Display) ja sowieso nur auf die öffentlichen REST API Zugriff hat. (2014-11-14)


In the long term it would be desirable to have the following user roles: Admin, Operator, Evaluator, DisplayApplication.



\subsubsection{REST interface}

Defining the REST API. 

% TODO: Explain why I chose which level of separation / detail.

Notes from before I started writing
\begin{enumerate}
\item Think about using a Extreme Programming approach http://www.extremeprogramming.org/rules.html
\end{enumerate}
\label{sec:implementation:modeling}

Modeling
	Software Model
	+ Dependencies in between them
	Users
	User Stories
	REST API








% \clearpage
% \subsection{Implementation}
\label{4d_implementation}


Briefly describe how I approached the implementation.

\begin{enumerate}
\item Challenges
\item Problems
\item ...
\end{enumerate}


\subsubsection{Package Manager}

\begin{enumerate}
\item NPM and Bower (instead of Browserify) + state reasons
\item State reasons, why I did / didn't check in the code for npm / bower components: \url{http://addyosmani.com/blog/checking-in-front-end-dependencies/}

    \begin{itemize}
    \item "prevent bad dependencies from breaking their app"
    \item "the longevity of package managers and their tooling"
    \item to be independent of other services and thus to garantee a longer life for the tool
    \end{itemize}
\end{enumerate}



\subsubsection{Frontend Framework}

give an overview of current Frontend Frameworks and briefly state which one I used why. A good comparision can be found here \url{http://www.sitepoint.com/5-most-popular-frontend-frameworks-compared/}

\begin{enumerate}
\item Bootstrap
\item Foundation by Zurb
\item ...
\end{enumerate}



\subsubsection{Deployment}

Practical deployment, best practices, how I proceded, and why

\begin{enumerate}
\item GitHub Repo
\item Deployment to Heroku
\end{enumerate}

\subsection{Not sure about}

	\begin{enumerate}

	\item \textbf{4.4 Implementation}
		\item leave out!!
		\item these aspets are all part of the MKDOCS documentation inside the repository!

	\item \textbf{Maybe:}
		\item challenges		>> for presentation
		\item problems		>> presentation or docu
		\item frameworks 		>> documentation
		\item deployment		>> documentation

	\end{enumerate}
