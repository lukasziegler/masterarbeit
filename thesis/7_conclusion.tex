\section{Conclusion}
\label{chapter:conclusion}
% inspired by http://www.wlu.ca/forms/1676/Conclusion.pdf

	% 1] What has been done
	Evaluating interactive applications on public display installations is as crucial as the development process itself. In this thesis we gave an overview of how other public display applications have been evaluated in literature, presented a categorization of standardized questionnaires, and introduced the \textit{PDSurvey} platform. This survey platform allowed us to assess our research questions through a lab study. Our main research questions were which feedback channel is best suited for completing surveys in public and what motivated our users to participate. 

	In the field study we offered the users four feedback channels to respond to the questionnaire. The options allowing users to respond directly in-situ were most popular. However the tablet turned out to be more popular than the primary display (TV screen). The tablet was preferred due to its smaller form factor, better usability, and because responding didn't feel as public. Despite the additional effort for responding via smartphone or email, these feedback channels were still an option for some. Reasons stated for the smartphone were personal possession and habit, for email because of having more time, being able to do it at home, and better warranty of privacy. It is interesting to see that around a fifth of the participants chose an indirect and more time consuming option, even though they had the opportunity to use low effort input devices such as the tablet or the TV screen.

	% 2] Our findings show ...
	Our field study has shown that there is an area of application for surveys being conducted on public displays and that this approach can simplify and support the evaluation of interactive applications. Of importance is a fast interaction time and low-effort input technique. When scaling this approach to large display networks, the evaluation process of new interactive applications can be simplified. In order to gain good insights into why certain effects and differences arise in public display setups, it is vital to assess detailed information about the context of each application, allowing us to determine which characteristics cause certain effects. This aspect is also the point which makes our platform unique. 

	% 3] Relevance, put the results into context
	We expect the number of interactive public display applications to increase in the future, and herewith also the demand for fast and easy evaluation of such. Why not utilize the interactive capability of todays public displays and use it as a direct feedback channel for quantitative evaluation? When running large display networks, this can be the first step towards a better understanding of the displays surrounding the environments and a faster problem analysis.


	% Emphasize the importance >> place it in a larger context!
	With this thesis we gained first insights into which feedback channels are suited for which context. We delivered a proof-of-concept for the evaluation of public displays to be executed on the displays themselves.
	The response rates were good (10\%), especially since the attraction of participants was solely based on intrinsic motivation. We came to the conclusion, that using public displays for assisting in the collection of survey data is a viable approach and worthy for further research.
	% 4) Summarize / conclude
	With our work we addressed the issue of time-consuming public display evaluation and contributed to the systematic evaluation of public display setups. With the overview of standardized questionnaires for public display evaluation we hope to bring a benefit to research community.




	% state, that this solves the issue

	% conclude in such a way, that the reader feels satisfied !!!

	% make it clear to the reader, that you have kept on track and supported your argument!!! (verify your goals from the abstract)

	% DO NOT end on a cliff hanger! don't leave the reader unsatisfied