
\section{Future Work}
\label{sec:future-work}

	% get inspired by: http://publications.lib.chalmers.se/records/fulltext/131908.pdf
	

\paragraph{Survey Platform}

	\begin{enumerate}
	\item Logging, adding more data sources (for tracking of ``User performance'', see chapter \ref{sec:questionnaires})
	\item Support the logging of video feeds. One possible approach would be to save the raw data in a dropbox account, to submit the file names via REST calls, and to access the files from PDSurvey via the Dropbox API
	\item Dynamically evaluating \textit{reliability} and \textit{validity} in the platform, ideally on-the-fly.
	\item also for pervasive displays?
	\end{enumerate}


\paragraph{Evaluation}

	\begin{enumerate}
	\item Number of questions tollerated on each evaluation channel (public display, tablet, smartphone, laptop/desktop)
	\item + also see chapter 5, subsection ``Research Questions'' 
	\item to also impose variables: context, content, location / setting and to vary those in experimental studies
	\item splitting 
	\item finding the similarities between different standardized questionnaires. As Jacucci et al. \cite{jacucci2010worldsofinformation} mentioned, there are often significant similarities between standardized questionnaires. It might be possible to break down each questionnaire to its principal parts, to bundle all conformities, in order to reduce the total amount of questions and to be able to split all questions across multiple users on the same display.
	\end{enumerate}


\paragraph{Other}
	\begin{enumerate}
	\item How should a survey be constructed to take best advantage of the PDSurvye platform? How many quantitative and how many qualitative questions?
	\end{enumerate}







\section{Conclusion}
\label{sec:conclusion}

	Outlook, be inspired by \url{http://www.wlu.ca/forms/1676/Conclusion.pdf}

		\begin{enumerate}
		\item which feedback channel:
		\item which question types: 
		\item quantitative vs qualitative: 
		\end{enumerate}


