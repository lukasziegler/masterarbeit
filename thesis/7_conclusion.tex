\section{Conclusion}
\label{chapter:conclusion}
% inspired by http://www.wlu.ca/forms/1676/Conclusion.pdf


	\begin{enumerate}
	\item which feedback channel:
	\item which question types: 
	\item quantitative vs qualitative: 
	\end{enumerate}


	Evaluating interactive applications on public display installations is as crucial as the development process itself. In this thesis we gave an overview of how other public display applications have been evaluated in literature, presented a categorization of standardized questionnaires, and introduced the \textit{PDSurvey} platform. 

	The design and development of the survey platform allowed us to validate our research questions through a lab study. Our main research questions were which feedback channel is best suited for completing surveys in public and what motivated our users to participate. 
	% In order to better understand how users respond to interactive questionnaires on public displays, we carried out a field study.

		For our display setup during the field study we offered four feedback channels. The options with direct response were most popular, however the tablet dominated the primary display (TV screen). The tablet was preferred due to its smaller form factor, better usability, and because responding didn't feel as public. Responding via smartphone or later via email were still chosen. The reasons stated for the smartphone because of the personal possession and habit, the email because of having more time, being able to do it at home, and better warranty of privacy. It is interesting to see that nonetheless over a fifth of the participants chose an indirect and more time consuming option, even though they had the opportunity to use low effort input devices such as the tablet or the TV screen.


	% Our findings show ...
	Our field study has shown that there is an area of application for surveys being conducted on public displays and that this approach can simplify and improve the evaluation of interactive applications. When scaling this approach to large display networks, the evaluation process of new interactive applications can be simplified.
	% When taking this to large scale, when deploying interactive applications to hundreds of public displays, an evaluation framework like PDSurvey is of need.

	In order to gain good insights into why certain effects and differences arise in public display setups, it is vital to assess the detailed context of each application, allowing us to determine which characteristics cause certain effects. This aspect is the point, which makes our platform unique. 

	We expect the number of interactive public display applications to increase in the future, and herewith also the demand for fast and easy evaluation. 
	% - When being able to utilize the interactive capability of 
	Why not utilize the interactive capability of todays public displays and use it as a direct feedback channel for quantitative evaluation? When running large display networks, this can be the first step towards an fast and effective problem analysis.


	% 3) Suggestions for the future

	% 4) Summarize / conclude
	>> conclude in such a way, that the reader feels satisfied, 

	Since we only scratched the surface with our research questions, it would be good to also look at other research questions (see chapter \ref{chapter:future-work}) regarding public display evaluation.

	MAAAEEEEHH, not happy with the last sentence yet
















	\vfill

	\textbf{NOTIZEN}

	% 1) Implicitly restate your thesis / position:
	es lohnt sich 
	quantitative Daten lassen sich automatisiert erheben
	die MOTIVATION war ...

	We showed that 
	and delivered a proof-of-concept for the evaluation of public displays to be executed on the displays themselves.

	Manko: privacy, social desirability, 
		% make it clear to the reader, that you have kept on track and supported your argument
		% verify that I established all, what I planned to do in the Abstract/Intro


	% 2) Emphasize the importance >> place it in a larger context!

	the thesis was important, because it 

	> (showed a proof-of-concept)
	> gained first insights into which feedback channels are suited for which context.


	RESULTS, OUR CONTRIBUATION
	Our findings show ...
	\begin{enumerate}
	\item > contribution: with the categorization of research questions / standardized questionnaires
	\item > interaction has to be fast, ideally a single-click interaction per question asked
	\item > we assume that the size of the public display has a direct proportional influence on the level of privacy and security. This however only is an assumption, which needs to be further investigated.
	\end{enumerate}

	\textbf{Concerns} that have been expressed: social desirability, less privacy
	These concerns should be taken into consideration when designing questionnaires, whether the benefit is larger than the drawback. Benefits being: direct response, no procrastination, immediate feedback of the users impressions, less people needed for evaluation, semi-automated evaluation of the log data.


	% 3) Suggestions for the future

	\begin{enumerate}
	\item context-based evaluation, across a multitude of displays, allowing us to recognize patterns
	\item include other data sources
	\item say why it is relevant: There is a ever increasing amount of displays, currently only a few interactive applications for these displays, assessing user satisfaction and other metrics is a time and resource intensive task, 
	\end{enumerate}

	% 4) Summarize / conclude
	conclude in such a way, that the reader feels satisfied, 

	maybe with an anecdote
	end in a concise and meaningful way




	note: a low effort input device is important,

	% DO NOT end on a cliff hanger! don't leave the reader unsatisfied