\section{Documentation}

\subsection{User Documentation}

"Most likely others will use your program. Writing a good user's manual will facilitate the use of your program. The important thing is to write for the naive user. It is best to assume that users of your program will know nothing about computers or their interfaces. A clear, concise, step-by-step description of how one uses your program can be of great value not only to others, but to you as well. You can identify awkward or misleading commands, and by correcting these, develop a much more usable product. Start from your requirements document to remind yourself what your program does."

\subsection{Developer Documentation}

\subsubsection{Authentication}

\begin{enumerate}
\item Login mechanisms: good explanation of the differences for Single-Page Applications (SPA) = \url{https://vickev.com/#!/article/authentication-in-single-page-applications-node-js-passportjs-angularjs}
\item + for more information, see Weeks 14 + 16 + 17
\end{enumerate}


During the prototyping phase we also 

For authentication there are the following common approaches, suited for RESTful web services.

\begin{itemize}
\item Token-based authentication: JSON Web Token (JWT)
\item Cookie-based authentication
\item 
\end{itemize}



\subsection{Maintenance Documentation}

"If your work has lasting benefit, someone will want to extend the functionality of your code. A well thought-out maintenance manual can assist in explaining your code. The maintenance manual grows from your specification, preliminary design, and detailed design documents. The manual shows how your program is decomposed into modules, specifies the interfaces between modules, and lists the major data structures and control structures. It should also specify the effective scope of changes to your code."
